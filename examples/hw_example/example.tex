\documentclass[10pt]{extarticle}
% I. Font encoding and choice
\usepackage[utf8]{inputenc}
\usepackage[T1]{fontenc}

% To use sans-serif fonts throughout
\usepackage{sansmathfonts}
\renewcommand*\familydefault{\sfdefault}

% II. Package imports and settings
\usepackage{
    geometry,
    amsmath,
    amssymb,
    amsfonts,
    amsthm,
    enumitem,
    fancyhdr,
    titlesec,
    parskip,
    bbm,
    listings,
    color,
    xcolor,
    caption,
    subcaption,
    etoolbox,
    booktabs,
    mathtools,
    graphicx,
    hyperref,
    wrapfig,
    lastpage,
    minted,
    tcolorbox
}

% Package `geometry`: set margins
\geometry{
    top=2.5cm,
    bottom=2.5cm,
    left=2.5cm,
    right=2.5cm
}

% Package `hyperref`: setup
\hypersetup{
    colorlinks,
    citecolor=blue,
    filecolor=blue,
    linkcolor=blue,
    urlcolor=blue,
}

% Package `fancyhdr`: setup
\makeatletter
\newcommand\email[1]{\def\@email{#1}}  % creates `\email{...}` topmatter
\newcommand\idnum[1]{\def\@idnum{#1}}  % creates `\idnum{...}` topmatter
\pagestyle{fancy}
\lhead{\@author}
\chead{\@title}
\rhead{\@date}
\lfoot{\@email}
\cfoot{Page \thepage{} of \hypersetup{linkcolor=.}\pageref{LastPage}}
\rfoot{\@idnum}
\makeatother

% Package `minted`: setup
\usemintedstyle{manni}

% Package `tcolorbox`: setup, `answer` and `ans` environment declarations
\tcbuselibrary{many}

% III. Custom symbols and commands
% Math symbols
\newcommand{\ints}{\mathbbmss{Z}} % Integers
\newcommand{\reals}{\mathbbmss{R}} % Reals
\newcommand{\nats}{\mathbbmss{N}} % Naturals
\newcommand{\rats}{\mathbbmss{Q}} % Rationals
\newcommand\iid{\stackrel{\mathclap{\normalfont\mbox{\tiny{\text{iid}}}}}{\sim}} % i.i.d. symbol
\newcommand\indep{\perp \!\!\!\! \perp} % Independence symbol
\newcommand\expect{\mathbbmss{E}\bracks*} % Expectation operator
\newcommand\prob{\mathbbmss{P}\parens*} % Probability operator
\newcommand\var{\text{Var}\parens*} % Variance operator
\newcommand\vvar{\mathbbmss{V}\parens*} % Alternate variance operator
\newcommand\cov{\text{Cov}\parens*} % Covariance operator
\DeclareMathOperator*{\argmax}{arg\,max} % Argmax
\DeclareMathOperator*{\argmin}{arg\,min} % Argmin

% Paired delimiters
\DeclarePairedDelimiter\ceil{\lceil}{\rceil}
\DeclarePairedDelimiter\floor{\lfloor}{\rfloor}
\DeclarePairedDelimiter\parens{\lparen}{\rparen}
\DeclarePairedDelimiter\bracks{\lbrack}{\rbrack}
\DeclarePairedDelimiter\braces{\lbrace}{\rbrace}
\DeclarePairedDelimiter\abs{\lvert}{\rvert}

% Differential operator
\renewcommand{\d}[1]{\ensuremath{\operatorname{d}\!{#1}}}

% Macro to specify the distribution of for a sequence of random variables
%   Ex 1: `\distn{X}{k}{Poisson}{\mu}` ==> `X_1, \ldots, X_k \sim \text{Poisson}{\mu}`
%   Ex 2: `\distn[\iid]{X}{k}{Poisson}{\mu}` ==> `X_1, \ldots, X_k \iid \text{Poisson}{\mu}`
% Where the `\iid` macro is defined as above
\newcommand{\distn}[5][\sim]{\ensuremath{
    {#2}_1, \ldots, {#2}_{#3}~{#1}~\text{{#4}}\left({#5}\right)
}}

\newcommand{\bs}{\boldsymbol} % Abbreviates `\boldsymbol`
\newcommand{\bsmc}[1]{\bs{\mathcal{#1}}} % Equivalent to `\boldsymbol{\mathcal{...}}`
\newcommand{\bsov}[1]{\bs{\overline{#1}}} % Equivalent to `\boldsymbol{\overline{...}}`

% IV. Custom environments
% Custom theorem environments from package `amstthm`
\newtheorem{theorem}{Theorem}
\newtheorem{corollary}{Corollary}[theorem]
\newtheorem{lemma}[theorem]{Lemma}
\newtheorem*{conjecture}{Conjecture}
\newtheorem*{definition}{Definition}

\theoremstyle{definition}
\newtheorem{question}{Question}
\newtheorem{problem}{Problem}
\newtheorem*{example}{Example}
\newtheorem{exercise}{Exercise}

\theoremstyle{remark}
\newtheorem*{remark}{Remark}
\newtheorem*{claim}{Claim}
\newtheorem*{solution}{Solution}

\newtheorem{myexample}{Example}
\newenvironment{fexample}
  {\begin{tcolorbox}\begin{fexample}}
  {\end{fexample}\end{tcolorbox}}


% `answer` box: more customizable
\newtcbtheorem[auto counter, number within = none]{answer}{Answer}{
    breakable,
    % color options!
    % a burnt orange...
    % colframe=red!75!yellow!75!black, colback = red!45!yellow!20!,
    % a robust maroon...
    % colframe=red!45!black, colback = red!10,
    % a royal purple...
    % colframe=red!55!blue!55!black, colback=red!7!blue!7!,
    % or blue.
    colframe=blue!30!black, colback=blue!10,
    fonttitle=\bfseries}{x}

% `ans` box: better for the basic use case
\newenvironment{ans}
{%
    \begin{answer*}[parbox=false]{}
    }{%
    \end{answer*}
}

% V. Miscellaneous preferences
% Set head/foot rule widths
\renewcommand{\headrulewidth}{0.0pt}
\renewcommand{\footrulewidth}{0.0pt}

% Solid black QED tombstone
\renewcommand\qedsymbol{$\blacksquare$}

% Allow augmented matrix
\makeatletter
\renewcommand*\env@matrix[1][*\c@MaxMatrixCols c]{%
  \hskip -\arraycolsep
  \let\@ifnextchar\new@ifnextchar
  \array{#1}}
\makeatother

% Number-within preferences
% \numberwithin{question}{section}
\numberwithin{equation}{question}

% Allow `tcolorbox` and other environments to run over page breaks
\allowdisplaybreaks{}

\title{DEPT 10100: Homework 1}
\author{First Last}
\date{\today}
\email{first.last@university.edu}
\idnum{School ID \#: 000001}


\begin{document}


% Question environment example
\begin{question}[10pts]
    Let $X_1 \sim \text{Gamma}(r_1, \lambda)$ and $X_2 \sim \text{Gamma}(r_2, \lambda)$ be independent random variables, and let $Y = X_1 + X_2$ and let $Z = X_1 / (X_1 + X_2)$.
    Find the joint density of $Y$ and $Z$, and find the marginal densities of $Y$ and $Z$. Identify the distributions of $Y$ and $Z$. Note that $\text{Gamma}(r, \lambda)$ distribution has density function
        \begin{equation*}
            f(x)=\frac{\lambda^{r}}{\Gamma(r)} e^{-\lambda x} x^{r-1}, \quad x>0
        \end{equation*}
\end{question}
% Answer environment example
\begin{ans}
    Since $X_1 \indep X_2$, their joint density is the product of their marginal densities.
    \begin{equation}
        f_{X_1, X_2}(x_1, x_2) = \frac{\lambda^{r_1 + r_2}}{\Gamma(r_1) \cdot \Gamma(r_2)}\cdot  e^{-\lambda(x_1 + x_2)} \cdot x_1^{r_1 - 1} \cdot x_2^{r_2 - 1}
    \end{equation}
    Let $X = [X_1, X_2]^\intercal$. Define
    \begin{equation}
        \begin{bmatrix}
            Y \\
            Z \\
        \end{bmatrix} = H(X) = \begin{bmatrix}
            X_1 + X_2 \\
            \frac{X_1}{X_1 + X_2}
        \end{bmatrix}
    \end{equation}
    The transformation $H(X)$ has Jacobian matrix
    \begin{equation}
        J(H) = \begin{bmatrix}
            1 & 1 \\
            \frac{X_2}{(X_1 + X_2)^2} & \frac{-X_1}{(X_1 + X_2)^2}
        \end{bmatrix}
    \end{equation}
    the determinant of which is $-1 / (X_1 + X_2)$. So
    \begin{align}
        F_{Y, Z}(y, z) &= \sum_{X : H(X) = [y, z]} \frac{\lambda^{r_1 + r_2}}{\Gamma(r_1) \cdot \Gamma(r_2)}\cdot  e^{-\lambda(x_1 + x_2)} \cdot x_1^{r_1 - 1} \cdot x_2^{r_2 - 1} \cdot \underbrace{(x_1 + x_2)}_{=1 / \abs{\det(J(H))}} \\
                       &= \frac{\lambda^{r_1 + r_2}}{\Gamma(r_1) \cdot \Gamma(r_2)} \cdot e^{-\lambda(x_1 + x_2)} \cdot x_1^{r_1} \cdot x_2^{r_2} \cdot \frac{x_1 + x_2}{x_1 x_2} \\
                       &= \frac{\lambda^{r_1 + r_2}}{\Gamma(r_1) \cdot \Gamma(r_2)} \cdot e^{-\lambda y} \cdot (yz)^{r_1} \cdot (y(1 -z))^{r_2} \cdot \frac{1}{yz(1-z)} \\
                       &= \underbrace{\frac{\lambda^{r_1 + r_2}}{\Gamma(r_1 + r_2)} \cdot e^{-\lambda y} \cdot y^{r_1 + r_2 - 1}}_{\text{Gamma}(r_1 + r_2, \lambda)} \cdot \underbrace{\frac{\Gamma(r_1 + r_2)}{\Gamma(r_1) \cdot \Gamma(r_2)} \cdot z^{r_1 - 1} (1 - z)^{r_2 - 1}}_{\text{Beta}(r_1, r_2)} \label{eqn:1a1}
    \end{align}
    So~\ref{eqn:1a1} is the joint probability density of $Y$ and $Z$. Then, we integrate out $y$ and $z$ to obtain their marginal densities. First, note that $y$ has support in $[0, \infty)$, whereas $z$ has support in $(0, 1)$. Next, as the underbraced portions of~\ref{eqn:1a1} note, the joint probability density of $Y$ and $Z$ is the product of two expressions, one in $y$, the other in $z$, each of which is a known probability density. So
    \begin{align}
        f_Y(y) &= \frac{\lambda^{r_1 + r_2}}{\Gamma(r_1 + r_2)} \cdot e^{-\lambda y} \cdot y^{r_1 + r_2 - 1} \cdot \int_0^1 \frac{\Gamma(r_1 + r_2)}{\Gamma(r_1) \cdot \Gamma(r_2)} \cdot z^{r_1 - 1} (1 - z)^{r_2 - 1} dz \\
               &= \frac{\lambda^{r_1 + r_2}}{\Gamma(r_1 + r_2)} \cdot e^{-\lambda y} \cdot y^{r_1 + r_2 - 1}
    \end{align}
    and
    \begin{align}
        f_Z(z) &= \frac{\Gamma(r_1 + r_2)}{\Gamma(r_1) \cdot \Gamma(r_2)} \cdot z^{r_1 - 1} (1 - z)^{r_2 - 1} \cdot \int_0^\infty \frac{\lambda^{r_1 + r_2}}{\Gamma(r_1 + r_2)} \cdot e^{-\lambda y} \cdot y^{r_1 + r_2 - 1} dy \\
               &= \frac{\Gamma(r_1 + r_2)}{\Gamma(r_1) \cdot \Gamma(r_2)} \cdot z^{r_1 - 1} (1 - z)^{r_2 - 1}
    \end{align}
    So $Y \sim \text{Gamma}(r_1 + r_2, \lambda)$ and $Z \sim \text{Beta}(r_1, r_2)$. Furthermore, since $f_{Y, Z}(y, z) = f_{Y}(y) \cdot f_Z(z)$, $Y \indep Z$.
\end{ans}

% Code example
\begin{minted}[breaklines, linenos]{python}
def egyptian_multiplication(a, n):
    """
    Returns the product `a * n`.
    Assume n is a nonegative integer
    """

    def is_odd(n):
        """
        Returns True if n is odd.
        """
        return n & 0x1 == 1

    if n == 1:
        return a
    if n == 0:
        return 0

    if is_odd(n):
        return egyptian_multiplication(a + a, n // 2) + a
    else:
        return egyptian_multiplication(a + a, n // 2)
\end{minted}


% Todo box example
\todo

% Todo box example with optional argument
\todo[message]


\end{document}
