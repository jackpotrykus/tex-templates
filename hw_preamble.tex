% I. Font encoding and choice
\usepackage[utf8]{inputenc}
\usepackage[T1]{fontenc}

% To use sans-serif fonts throughout
\usepackage{sansmathfonts}
\renewcommand*\familydefault{\sfdefault}

% II. Package imports and settings
\usepackage{
    geometry,
    amsmath,
    amssymb,
    amsfonts,
    amsthm,
    enumitem,
    fancyhdr,
    titlesec,
    parskip,
    bbm,
    listings,
    color,
    xcolor,
    caption,
    subcaption,
    etoolbox,
    booktabs,
    mathtools,
    graphicx,
    hyperref,
    wrapfig,
    lastpage,
    minted,
    tcolorbox
}

% Package `geometry`: set margins
\geometry{
    top=2.5cm,
    bottom=2.5cm,
    left=2.5cm,
    right=2.5cm
}

% Package `hyperref`: setup
\hypersetup{
    colorlinks,
    citecolor=blue,
    filecolor=blue,
    linkcolor=blue,
    urlcolor=blue,
}

% Package `fancyhdr`: setup
\makeatletter
\newcommand\email[1]{\def\@email{#1}}  % creates `\email{...}` topmatter
\newcommand\idnum[1]{\def\@idnum{#1}}  % creates `\idnum{...}` topmatter
\pagestyle{fancy}
\lhead{\@author}
\chead{\@title}
\rhead{\@date}
\lfoot{\@email}
\cfoot{Page \thepage{} of \hypersetup{linkcolor=.}\pageref{LastPage}}
\rfoot{\@idnum}
\makeatother

% Package `minted`: setup
\usemintedstyle{manni}

% Package `tcolorbox`: setup, `answer` and `ans` environment declarations
\tcbuselibrary{many}

% III. Custom symbols and commands
% Math symbols
\newcommand{\ints}{\mathbbmss{Z}} % Integers
\newcommand{\reals}{\mathbbmss{R}}                                                 % Reals
\newcommand{\nats}{\mathbbmss{N}}                                                  % Naturals
\newcommand{\rats}{\mathbbmss{Q}}                                                  % Rationals
\DeclareMathOperator*{\argmax}{arg\,max}                                           % Argmax opertaor
\DeclareMathOperator*{\argmin}{arg\,min}                                           % Argmin operator
\newcommand\iid{\stackrel{\mathclap{\normalfont\mbox{\tiny{\text{iid}}}}}{\sim}}   % i.i.d. symbol
\newcommand\indep{\perp \!\!\!\! \perp}                                            % Independence symbol
\newcommand\expectsym{\mathbbmss{E}}                                               % Expectation symbol
\newcommand\probsym{\mathbbmss{P}}                                                 % Probability symbol
\DeclareMathOperator*{\varsym}{Var}                                                % Variance symbol
\DeclareMathOperator*{\covsym}{Cov}                                                % Covariance symbol
\newcommand\prob{\probsym\parens*}                                                 % Probability operator
\newcommand\expect{\expectsym\bracks*}                                             % Expectation operator
\newcommand\var{\varsym\parens*}                                                   % Variance operator with ()
\newcommand\vvar{\mathbbmss{V}\parens*}                                            % Alternate variance operator
\newcommand\cov{\covsym\parens*}                                                   % Covariance operator

% Paired delimiters
\DeclarePairedDelimiter\ceil{\lceil}{\rceil}
\DeclarePairedDelimiter\floor{\lfloor}{\rfloor}
\DeclarePairedDelimiter\parens{\lparen}{\rparen}
\DeclarePairedDelimiter\bracks{\lbrack}{\rbrack}
\DeclarePairedDelimiter\braces{\lbrace}{\rbrace}
\DeclarePairedDelimiter\abs{\lvert}{\rvert}

% Differential operator
\renewcommand{\d}[1]{\ensuremath{\operatorname{d}\!{#1}}}

% Macro to specify the distribution of for a sequence of random variables
%   Ex 1: `\distn{X}{k}{Poisson}{\mu}` ==> `X_1, \ldots, X_k \sim \text{Poisson}{\mu}`
%   Ex 2: `\distn[\iid]{X}{k}{Poisson}{\mu}` ==> `X_1, \ldots, X_k \iid \text{Poisson}{\mu}`
% Where the `\iid` macro is defined as above
\newcommand{\distn}[5][\sim]{\ensuremath{
    {#2}_1, \ldots, {#2}_{#3}~{#1}~\text{{#4}}\left({#5}\right)
}}

\newcommand{\bs}{\boldsymbol} % Abbreviates `\boldsymbol`
\newcommand{\bsmc}[1]{\bs{\mathcal{#1}}} % Equivalent to `\boldsymbol{\mathcal{...}}`
\newcommand{\bsov}[1]{\bs{\overline{#1}}} % Equivalent to `\boldsymbol{\overline{...}}`
\newcommand{\bswh}[1]{\bs{\widehat{#1}}} % Equivalent to `\boldsymbol{\mathcal{...}}`

% IV. Custom environments
% Custom theorem environments from package `amstthm`
\newtheorem{theorem}{Theorem}
\newtheorem{corollary}{Corollary}[theorem]
\newtheorem{lemma}[theorem]{Lemma}
\newtheorem*{conjecture}{Conjecture}
\newtheorem*{definition}{Definition}

\theoremstyle{definition}
\newtheorem{question}{Question}
\newtheorem{problem}{Problem}
\newtheorem*{example}{Example}
\newtheorem{exercise}{Exercise}

\theoremstyle{remark}
\newtheorem*{remark}{Remark}
\newtheorem*{claim}{Claim}
\newtheorem*{solution}{Solution}

\newtheorem{myexample}{Example}
\newenvironment{fexample}
  {\begin{tcolorbox}\begin{fexample}}
  {\end{fexample}\end{tcolorbox}}


% `answer` box: more customizable
\newtcbtheorem[auto counter, number within = none]{answer}{Answer}{
    breakable,
    % color options!
    % a burnt orange...
    % colframe=red!75!yellow!75!black, colback = red!45!yellow!20!,
    % a robust maroon...
    % colframe=red!45!black, colback = red!10,
    % a royal purple...
    % colframe=red!55!blue!55!black, colback=red!7!blue!7!,
    % or blue.
    colframe=blue!30!black, colback=blue!10,
    fonttitle=\bfseries}{x}

% `ans` box: better for the basic use case
\newenvironment{ans}
{%
    \begin{answer*}[parbox=false]{}
    }{%
    \end{answer*}
}

% Todo-box. Put a todo-box in your document. It's big and red, and you won't miss it.
% Use `\todo` to put a box with just "Todo" in it,
% use `\todo[description]` to put a box with "Todo: description" in it.
\newcommand{\todo}[1][]{%
    \begin{tcolorbox}[
            hbox,
            colback  = red!20,
            colframe = red!45,
        ]
            \ifstrequal{#1}{}
            { \bfseries\Large\color{red} Todo }
            { \bfseries\Large\color{red} Todo: #1 }
    \end{tcolorbox}
}


% V. Miscellaneous preferences
% Set head/foot rule widths
\renewcommand{\headrulewidth}{0.0pt}
\renewcommand{\footrulewidth}{0.0pt}

% Solid black QED tombstone
\renewcommand\qedsymbol{$\blacksquare$}

% Allow augmented matrix
\makeatletter
\renewcommand*\env@matrix[1][*\c@MaxMatrixCols c]{%
  \hskip -\arraycolsep
  \let\@ifnextchar\new@ifnextchar
  \array{#1}}
\makeatother

% Number-within preferences
% \numberwithin{question}{section}
\numberwithin{equation}{question}

% Allow `tcolorbox` and other environments to run over page breaks
\allowdisplaybreaks{}
